\documentclass[journal]{IEEEtran}
\usepackage{cite}
\hyphenation{op-tical net-works semi-conduc-tor}
\usepackage[utf8]{inputenc}
\begin{document}

\title{Cyber Security Roadmap From Institutional Perspective}

\author{Muhammer KARAMAN, İbrahim ŞİŞANECİ% <-this % stops a space
\thanks{dip not XXXX}% <-this % stops a space
\thanks{dip not XXXX}% <-this % stops a space
\thanks{dip not XXXX}}

% note the % following the last \IEEEmembership and also \thanks -
% these prevent an unwanted space from occurring between the last author name
% and the end of the author line. i.e., if you had this:
%
% \author{....lastname \thanks{...} \thanks{...} }
%                     ^------------^------------^----Do not want these spaces!
%
% a space would be appended to the last name and could cause every name on that
% line to be shifted left slightly. This is one of those "LaTeX things". For
% instance, "\textbf{A} \textbf{B}" will typeset as "A B" not "AB". To get
% "AB" then you have to do: "\textbf{A}\textbf{B}"
% \thanks is no different in this regard, so shield the last } of each \thanks
% that ends a line with a % and do not let a space in before the next \thanks.
% Spaces after \IEEEmembership other than the last one are OK (and needed) as
% you are supposed to have spaces between the names. For what it is worth,
% this is a minor point as most people would not even notice if the said evil
% space somehow managed to creep in.


%%%%
% Replace this line with your paper title. -- Suat Ozdemir
%%%%
% The paper headers
\markboth{Cyber Security Roadmap From Institutional Perspective}%
{Shell \MakeLowercase{\textit{et al.}}: Bare Demo of IEEEtran.cls for Journals}


\maketitle


\begin{abstract}
%\boldmath
Nowadays, cyber world has become indispensable through digital technologies are used extensively in all aspects of our lives . Besides many advanteges/facilities coming with the cyber world, mankinds is facing new threads.In this cyberworld, in addition to the classic security needs, every member which is from corporations to individuals needs cyber security. A novel concept \"Insttitiaonal Cyber Security\" is suggested.This article shed light on cyber security institutions to ensure that there is a roadmap. In this article, a roadmap which will shed light on institutions to ensure cyber security is recommended . This roadmap against cyber threats contains cyber security elements such as policies, plans, awareness actities  and works including technical and administrative measures.

\end{abstract}
% IEEEtran.cls defaults to using nonbold math in the Abstract.
% This preserves the distinction between vectors and scalars. However,
% if the journal you are submitting to favors bold math in the abstract,
% then you can use LaTeX's standard command \boldmath at the very start
% of the abstract to achieve this. Many IEEE journals frown on math
% in the abstract anyway.

% Note that keywords are not normally used for peerreview papers.
\begin{IEEEkeywords}
Cyber Security,Information Security,
\end{IEEEkeywords}

\IEEEpeerreviewmaketitle



\section{Introduction}

%\hfill mds

%\hfill January 11, 2007

\IEEEPARstart{N}{owadays}....XXXX.
In this work, the word “institution” is used for public or private sector companies/corparations and state institution which have critical infrastractures.

%\subsection{Subsection Heading Here}
%Subsection text here.

% needed in second column of first page if using \IEEEpubid
%\IEEEpubidadjcol

%\subsubsection{Subsubsection Heading Here}
%Subsubsection text here.


% An example of a floating figure using the graphicx package.
% Note that \label must occur AFTER (or within) \caption.
% For figures, \caption should occur after the \includegraphics.
% Note that IEEEtran v1.7 and later has special internal code that
% is designed to preserve the operation of \label within \caption
% even when the captionsoff option is in effect. However, because
% of issues like this, it may be the safest practice to put all your
% \label just after \caption rather than within \caption{}.
%
% Reminder: the "draftcls" or "draftclsnofoot", not "draft", class
% option should be used if it is desired that the figures are to be
% displayed while in draft mode.
%
%\begin{figure}[!t]
%\centering
%\includegraphics[width=2.5in]{myfigure}
% where an .eps filename suffix will be assumed under latex,
% and a .pdf suffix will be assumed for pdflatex; or what has been declared
% via \DeclareGraphicsExtensions.
%\caption{Simulation Results}
%\label{fig_sim}
%\end{figure}

% Note that IEEE typically puts floats only at the top, even when this
% results in a large percentage of a column being occupied by floats.


% An example of a double column floating figure using two subfigures.
% (The subfig.sty package must be loaded for this to work.)
% The subfigure \label commands are set within each subfloat command, the
% \label for the overall figure must come after \caption.
% \hfil must be used as a separator to get equal spacing.
% The subfigure.sty package works much the same way, except \subfigure is
% used instead of \subfloat.
%
%\begin{figure*}[!t]
%\centerline{\subfloat[Case I]\includegraphics[width=2.5in]{subfigcase1}%
%\label{fig_first_case}}
%\hfil
%\subfloat[Case II]{\includegraphics[width=2.5in]{subfigcase2}%
%\label{fig_second_case}}}
%\caption{Simulation results}
%\label{fig_sim}
%\end{figure*}
%
% Note that often IEEE papers with subfigures do not employ subfigure
% captions (using the optional argument to \subfloat), but instead will
% reference/describe all of them (a), (b), etc., within the main caption.


% An example of a floating table. Note that, for IEEE style tables, the
% \caption command should come BEFORE the table. Table text will default to
% \footnotesize as IEEE normally uses this smaller font for tables.
% The \label must come after \caption as always.
%
%\begin{table}[!t]
%% increase table row spacing, adjust to taste
%\renewcommand{\arraystretch}{1.3}
% if using array.sty, it might be a good idea to tweak the value of
% \extrarowheight as needed to properly center the text within the cells
%\caption{An Example of a Table}
%\label{table_example}
%\centering
%% Some packages, such as MDW tools, offer better commands for making tables
%% than the plain LaTeX2e tabular which is used here.
%\begin{tabular}{|c||c|}
%\hline
%One & Two\\
%\hline
%Three & Four\\
%\hline
%\end{tabular}
%\end{table}


% Note that IEEE does not put floats in the very first column - or typically
% anywhere on the first page for that matter. Also, in-text middle ("here")
% positioning is not used. Most IEEE journals use top floats exclusively.
% Note that, LaTeX2e, unlike IEEE journals, places footnotes above bottom
% floats. This can be corrected via the \fnbelowfloat command of the
% stfloats package.

Information technology has become pervasive in every way—from our phones and other small devices to our enterprise networks to the infrastructure that runs our economy. Improvements to the security of this information technology are essential for our future. As the critical infrastructures of the United States have become more and more dependent on public and private networks, the potential for widespread national impact resulting from disruption or failure of these networks has also increased.

Cyberattacks are increasing in frequency and impact. Adversaries seeking to disrupt the nation’s critical infrastructures are driven by different motives and view cyberspace as a possible
means to have much greater impact, such as causing harm to people or widespread
economic damage. Although to date no cyberattack has had a significant impact on
our nation’s critical infrastructures, previous attacks have demonstrated that extensive vulnerabilities exist in information systems and networks, with the potential for serious damage. The effects of a successful attack might include serious economic consequences through impacts on major economic and industrial sectors, threats to infrastructure elements such as electric power, and disruptions that impede the
response and communication capabilities of first responders in crisis situations.

This cybersecurity R and D roadmap  is an attempt to begin to define a national R\&D agenda that is required to enable us to get ahead of our adversaries and produce the technologies that will protect our information systems and networks into the future. The research, development, test, evaluation, and other life cycle considerations required are far reaching—from technologies that secure individuals and their information to technologies that will ensure that our critical infrastructures are much more resilient. These investments must tackle the vulnerabilities of today and envision those of the future.



\section{Cyberspace And Cyber Threats }
* The term of security should be re-defined in terms of cyber security with including the components of cyber space and emerging cyber threats. What is more important  than bridging a new term between security and cyber security, is the perception and understanding of the individuals of  institutions about the new aspects of security definition or redefined security term.

Definition of cyberspace "the notional environment in which communication over computer networks occurs." \cite{OxfordDictionary:2013}

\section{Cyber Security Approach}

\subsection{Current Concepts}
Siber savunma, Siber Harekat vd.Bilgi teminatı    cyber defence, intelligence, operations
info sec compu sec , information assurance vs.
\subsubsection{Cybersecurity}

Definition of cybersecurity in the dictionary,"the state of being protected against the criminal or unauthorized use of electronic data, or the measures taken to achieve this"\cite{OxfordDictionary:2013}

To understand the term cybersecurity we must first define the term cyberrisk.

Cyberrisk is not one specific risk. It is a group of risks, which differ in technology, attack vectors, means, etc. We address these risks as a group largely due to two similar characteristics: A) they all have a potential great impact B) they were all once considered improbable.\cite{Barzilay:2013:ISACAOnline}
\subsubsection{Cyberrisk}

Cybersecurity is the sum of efforts invested in addressing cyberrisk, much of which was, until recently, considered so improbable that it hardly required our attention.

XXX Cyberrisk kurumlar için de geçerli..  
\subsubsection{Cyberdefence}
\subsubsection{CyberOperations}

\subsection{Development of Cyber Security}

Mevcut süreçler, ISO 27001, BSYG vb.
,  klısımda uluslaraso  ve ülke bazı oluştır
ihtiçi


\subsubsection{From Information Security to Cyber Security}

\subsubsection{A New Concept Institutional Cyber Security}

* Enforcement of cyber security rules and policies must be met by the institutions and auditing should be taken into consideration whether the  institutions comply with given set of cyber security policies.  Institutions should not have a choice not to be a part of cyber security enforcement process (Needs to be defined).

* The terms of INFOSEC and COMPUSEC are not providing cyber security needs of the institutions. Altough protecting institutional data, providing business continuity and so forth, The Process of Information Security Management  fails to envision, deter, mitigate or prevent some large scale and targetted malwares like fatmall, duqu, stuxnet, and so forth that in other terms what we call Advanced Persistant Threats. (Brief info about APT can be given here)

grafik çiz.

The ITU also defined cyber security broadly as:‘[T]he collection of tools, policies, security concepts, security safeguards, guidelines, risk management approaches, actions, training, best practices, assurance and technologies that can be used to protect the cyber environment and organization and user’s assets. Organization and user’s assets include connected computing devices, personnel, infrastructure, applications, services, telecommunications systems, and the totality of transmitted and/or stored information in the cyber environment. Cybersecurity strives to ensure the attainment and maintenance of the security properties of the organization and user’s assets against relevant security risks in the cyber environment. The general security objectives comprise the following: availability; integrity, which may include authenticity and non-repudiation; and confidentiality.’Many countries are defining what they mean by cyber security in their respective national strategy documents.
\cite{kurt2012cyber}
1.4.1. The Three Dimensions: Governmental, National and  International

Any approach to a NCS strategy needs to consider the ‘three dimensions’ of activity:
the governmental, the national (or societal) and the international. Since the 1990s a
particular trend in public policy theory has focused on the cooperation of different
actors. Initially the focus was on improving the coordination of government
actors (the Whole of Government approach or WoG), particularly between the
departments most involved in stabilisation or peace building operations in places like Afghanistan or Iraq.
\\


To follow a cyber security roadmap is crucial for the critical infrastructures like telecomunications, transportation, health sectors and so forth that form essence and  is part of national security (Referans, Bilge Hoca’nın Makalesi) other governmental sectors like agriculture, forestry that may seem less important in terms of being a criical infrastructure,  , non governmental and private sectors are also inevitably need to track a cyber security roadmap.
\\

\subsection{ DILEMMAS OF NATIONAL CYBER SECURITY}
THE FIVE DILEMMAS OF NATIONAL CYBER SECURITY
1.5.1. Stimulate the Economy vs. Improve National Security
1.5.2. Infrastructure Modernisation vs. Critical Infrastructure Protection
1.5.3. Private Sector vs. Public Sector
1.5.4. Data Protection vs. Information Sharing
1.5.5. Freedom of Expression vs. Political Stability\cite{kurt2012cyber} XXX Buradaki ikilemlerden 1  cümle ile değerlendirme kısmında ya da  policy kısmında değinilebilir.
\subsection{Challenges}
While the weakest chain in security is human being, in terms of national security the weakest chain is the weakest institution having critical infrastructures.
\\
Challenges
- Legal
- Leadership
- Cost
- Lack of trained personnel, human resource

Migration plan
XXX Outsurcing mi in home development mı?
IT security professionals said that outsourcing would be the biggest cybersecurity threat

\section{Institutional Cyber Security Roadmap}
Ulusal/kurumsal yol haritası Yapılacakların özeti
\subsection{Significance of Roadmapping}
* The significance of this work is to fill the absence of such kind of roadmaps and guidelines  dedicated solely to institutions.
\\

[1] Numarlı kaynaktan alınabilecek husular
The United States is at a significant decision point. We must continue to defend our current systems and networks and at the same time attempt to “get out in front” of our adversaries and ensure that future generations of technology will position us to better protect our critical infrastructures and respond to attacks from our adversaries.
are much more resilient.

The R\&D investments recommended in this roadmap must tackle the vulnerabilities of today and envision those of the future.

11 hard problems on cyber security of countries.
XXX Burada ABD kendisi ve ülkeler için zor problem olan sahaları /araştırma alanlarını sıralamış bu sahaların kurumlar için olanlar listelenebilir ve bu listelenen problemleri çözmeye aday bir yol haritası sunuyoruz diyebiliriz.
Each of the following topic areas is treated in detail in a subsequent section of its own, from Section 1 to Section 11.
1. Scalable trustworthy systems (including system architectures and requisite development methodology)
2. Enterprise-level metrics (including measures of overall system trustworthiness)
3. System evaluation life cycle (including approaches for sufficient assurance)
4. Combatting insider threats
5. Combatting malware and botnets
6. Global-scale identity management
7. Survivability of time-critical systems
8. Situational understanding and attack attribution
9. Provenance (relating to information, systems, and hardware)
10.Privacy-aware security
11.Usable security

 [1]

Why an institution neeeds a road map?

\subsection{Methodology}
The methodology for development an cyber security roadmap  is similar to technology roadmapping because cyberspace grows/changes/develop fast like technological devepments. The methodology for development an cyber security roadmap consist of three phases: preliminary activity, development of the roadmap and follow-up activity.\cite{maughan2009roadmap} 

In the Phase I. Preliminary activity
1. Satisfy essential conditions.
3. Define the scope and boundaries for the roadmap.

Phase II. Development of the Roadmap
1. Identify the “cyber security issues” that will be the focus of the roadmap.
2. Identify the critical cyber security requirements and their targets.
3. Specify the major cyber security areas.
4. Specify the cyber security drivers and their targets.
6. Recommend the technology alternatives that should be pursued.

Phase III. Follow-up activity
1. Critique and validate the roadmap.
2. Develop an implementation plan.
\\

	
	Makale de bulunmasını isteğimiz sahalar
	The identification and description of each technology area and its current status.
	• Critical factors (show-stoppers) which if not met will cause the roadmap to fail.
	• Areas not addressed in the roadmap.
	• Technical recommendations.
	• Implementation recommendations.

\subsection{XXX}

\subsubsection{Security perception \& Awareness}
* The relation between habits and security perception should be probed among individuals especially among system administrators.
\subsubsection{XXX}
At both the individual corporate and industry levels, technology roadmapping has
several potential uses and resulting benefits. Three major uses are:
• First, technology roadmapping can help develop a consensus about a set of needs and
the technologies required to satisfy those needs.
• Second, it provides a mechanism to help experts forecast technology developments in
targeted areas.
• Third, it can provide a framework to help plan and coordinate technology
developments both within a company or an entire industry.[51] XXX  restament

XXX 2 tür Roadmapping var .. biz bir  tanesiinin  üzerinden gidelim
This roadmapping consists of three phases:
1. Assessment (i.e., establish assumption, establish regulatory requirements, establish
committed milestones, depict logics and planned activities).
2. Analysis (i.e., identify issues, perform root-cause analysis, and translate issues to
activities).
3. Resolution (develop issue-resolution documents and integrate activities with activity
data sheets).
Although there are some similarities, this roadmapping approach is fundamentally
different (in purpose, scope, and steps) from the technology roadmapping process
addressed by this paper.\cite{garcia1997fundamentals}
\\

XXX 1 numaralı kaynağımızda research için bir yol haritası derken kısa orta ve uzun vadede neler yapılabileğini listelemiş.. ve milestonelar sıralamış Bizimki Roadmap olacaksa bizdede bu tür bir tasnif gerekli.
1. kaynakta her bir problem için TABLE 1.1: Summary of Gaps, Approaches, and Benefits isimli tablolarda sorunlar özetleniyor.. bunlardan kurumsal olanları süzebiliriz.\cite{maughan2009roadmap}

\section{Conclusion}
Burada başarılı olmak için gerekli olana ilave hususlar vs.
eklenebilir.
 Bizim yeniliğimiz
1. devlet kurumları ve büyük ölçekli firmalar için kurumsal yol haritası olması
2.
\\
Reccomendation
Bizim yol haritası kendileri için yol haritası hazırlamak isteyen kurumlar için genelbir yol.har. Kuurumlar yapılarına durumlarına özel mutlaka daha detay lı yh.ları hazırlamalısırlar.




% if have a single appendix:
%\appendix[Proof of the Zonklar Equations]
% or
%\appendix  % for no appendix heading
% do not use \section anymore after \appendix, only \section*
% is possibly needed

% use appendices with more than one appendix
% then use \section to start each appendix
% you must declare a \section before using any
% \subsection or using \label (\appendices by itself
% starts a section numbered zero.)
%


\appendices
\section{Proof of the First Zonklar Equation}
Appendix one text goes here.

% you can choose not to have a title for an appendix
% if you want by leaving the argument blank
\section{}
Appendix two text goes here.


% use section* for acknowledgement
\section*{Acknowledgment}


The authors would like to thank Turkish Armed Forces Cyber Defence Center staff.
% Can use something like this to put references on a page
% by themselves when using endfloat and the captionsoff option.
\ifCLASSOPTIONcaptionsoff
  \newpage
\fi



% trigger a \newpage just before the given reference
% number - used to balance the columns on the last page
% adjust value as needed - may need to be readjusted if
% the document is modified later
%\IEEEtriggeratref{8}
% The "triggered" command can be changed if desired:
%\IEEEtriggercmd{\enlargethispage{-5in}}

% references section

% can use a bibliography generated by BibTeX as a .bbl file
% BibTeX documentation can be easily obtained at:
% http://www.ctan.org/tex-archive/biblio/bibtex/contrib/doc/
% The IEEEtran BibTeX style support page is at:
% http://www.michaelshell.org/tex/ieeetran/bibtex/
%\bibliographystyle{IEEEtran}
% argument is your BibTeX string definitions and bibliography database(s)
%\bibliography{IEEEabrv,../bib/paper}
%
% <OR> manually copy in the resultant .bbl file
% set second argument of \begin to the number of references
% (used to reserve space for the reference number labels box)
\bibliographystyle{IEEEtran}
\bibliography{IEEEabrv,Master}


% biography section
%
% If you have an EPS/PDF photo (graphicx package needed) extra braces are
% needed around the contents of the optional argument to biography to prevent
% the LaTeX parser from getting confused when it sees the complicated
% \includegraphics command within an optional argument. (You could create
% your own custom macro containing the \includegraphics command to make things
% simpler here.)
%\begin{biography}[{\includegraphics[width=1in,height=1.25in,clip,keepaspectratio]{mshell}}]{Michael Shell}
% or if you just want to reserve a space for a photo:

\begin{IEEEbiography}{Michael Shell}
Biography text here.
\end{IEEEbiography}

% if you will not have a photo at all:
\begin{IEEEbiographynophoto}{John Doe}
Biography text here.
\end{IEEEbiographynophoto}

% insert where needed to balance the two columns on the last page with
% biographies
%\newpage

\begin{IEEEbiographynophoto}{Jane Doe}
Biography text here.
\end{IEEEbiographynophoto}

% You can push biographies down or up by placing
% a \vfill before or after them. The appropriate
% use of \vfill depends on what kind of text is
% on the last page and whether or not the columns
% are being equalized.

%\vfill

% Can be used to pull up biographies so that the bottom of the last one
% is flush with the other column.
%\enlargethispage{-5in}



% that's all folks
\end{document}
